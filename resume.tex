
%%% Local Variables:
%%% mode: latex
%%% TeX-master: t
%%% End:
% LaTeX file for resume
% This file uses the resume document class (res.cls)

\documentclass[margin]{res}
\usepackage{helvetica} % uses helvetica postscript font (download helvetica.sty)
%\usepackage{newcent}   % uses new century schoolbook postscript font
\topmargin=-0.5in  % start text higher on the page
\setlength{\textheight}{10in} % increase text height to fit resume on 1 page
\begin{document}
\name{Corbin Souffrant}

\address{ 1102 S Abel St Apt. 464, Milpitas CA, 95035 \\
  Website: www.corbinsouffrant.com \\ Email: souffra2@illinois.edu \\
  Phone: (650) 732-9455 \\ loliponi on Twitter/freenode/reddit/etc...
  }


\begin{resume}

\section{OBJECTIVE} Seeking a full time position. Willing to work in Computer Security Research,
Vulnerability Research and Development, or Malware Analysis.  I prefer sticking in Exploit Research.

\section{EDUCATION}       University of Illinois Urbana-Champaign \\
                Computer Science Engineering B.S., May 2014 \\
                GPA 3.19/4.0

                \begin{ncolumn}{2}
                {\bf Relevant Courses} \\
                CS461 - Computer Security I & CS460 - Computer Security Lab \\
                CS498SH - Malware Analysis Lab & CS498LA - Undergraduate Research Lab \\
                CS598MAN - Applied Cryptography & CS498AL1 - Digital Forensics Lab \\
                CS591RHC - Security Reading Group & CS598MCC - Network
                Security 
		\end{ncolumn}

\section{SKILLS}
\normalsize{\section{Security}}
                 \begin{itemize} \itemsep -2pt
                 \item Static and Dynamic Analysis: IDAPro, OllyDBG,
                   GDB, Wireshark
                 \item Vulnerability Discovery: Fuzzer Development,
                   Source Code Auditing, Testing Framework Design
                 \item Malware Analysis: Unpacking techniques,
                     Anti-debugging + Anti-Reversing removal
                 \end{itemize}
\normalsize{\section{Languages and Libraries}}
	           \begin{itemize} \itemsep -2pt % reduce space between items
                   \item Python, C/C++, x86, MIPS, Java
                   \item ROS (Robot Operating System), OpenCV, Android
                   \end{itemize}

\section{WORK EXPERIENCE}
\begin{tabular}{p{3in} r} % setup 2 columns, first
					    % is 3 inches wide
                 Fireeye - Research Engineer/Scientist &  June '14 - Present
                  \end{tabular}
                   \begin{itemize} \itemsep -2pt % \item[] prevents a bullet from appearing
                     \item Exploit Research team.
                     \item Monitored and extended functionality of the ZeroDay Discovery Center, and eventually
                     became the project owner.  It is an automated tool that gathers samples in the wild, and
                     analyzes them for possible zero days.
                     \item Developed new exploitation discovery techniques.  Helped implement tools that used 
                     innovative techniques in order to detect exploits as they occured.
                     \item Supported other teams with expert advice on exploits.  Was a point of contact on the
                     exploit team to answer questions and lead engineering teams on current exploit techniques,
                     in order to help them prioritize projects.
                     \item Analyzed possible zero days and other high priority malware samples that were either
                     caught by the ZeroDay Discovery Center or sent to me by other teams.  This involved extracting
                     ROP chains and payloads, as well as determining the cause of exploitation in order to report
                     to necessary companies.
		   \end{itemize}
\begin{tabular}{p{3in} r} % setup 2 columns, first
					    % is 3 inches wide
                  University of Illinois - Teaching Assistant&  Spring 14
                  \end{tabular}
                   \begin{itemize} \itemsep -2pt % \item[] prevents a bullet from appearing
                     \item Designed and helped teach the CS460 - Computer Security Lab course.
                     \item Rewrote the lectures and labs for CS460 as my senior thesis/project with the
                     help of John Bambenek.  This was a course of about 60 mostly junior/senior level 
                     students who had taken Computer Security I.  The goals of the course was to give them 
                     a hands on approach for Offensive (basic security flaws in web + systems, building ROP,
                     creating payloads) and Defensive (system administration, secure coding techniques, firewall/IDS
                     monitoring).  The course culminated in a team based Attack/Defense lab, where they attacked each
                     other and defended their own systems on a set of virtual machines.
		   \end{itemize}
\begin{tabular}{p{3in} r} % setup 2 columns, first
					    % is 3 inches wide
                  Raytheon SIGOVS - Intern &  Summer 13
                  \end{tabular}
\begin{itemize} \itemsep -2pt % \item[] prevents a bullet from appearing
                    \item Vulnerability Discovery using a variety of
                      methods.  Developed a smart fuzzer and setup a
                      testing framework.  Found bugs via source code
                      auditing.  Worked with browser security
                      and applications on both x86 and ARM architectures.
		   \end{itemize}
		 \begin{tabular}{p{3in} r}
                  University of Illinois - Researcher &  Spring - Summer 12
                 \end{tabular}
		  \begin{itemize} \itemsep -2pt
                   \item Worked with Professor Sam King to design
                     and implement an application framework for
                     general purpose robots.  This involved developing
                     an API to communicate with the robot via a web
                     and android application.  Also wrote applications
                     in python and C++.
                   \item Presented a poster for
                     the research at a research symposium in Siebel in
                     Spring 2012
                  \end{itemize}

\section{PROJECTS}
\begin{tabular}{p{3in} r} % setup 2 columns, first
Automated Malware Analysis & Spring 13
\end{tabular}
\begin{itemize} \itemsep -2pt
\item Set up a Virtual Machine that accepted binaries from a web
  interface.  I then used YARA and CuckooSandbox to process the binary
  and store the results in a database.  I worked with 3 other
  students for a semester project in the Security Lab course.
\end{itemize}
\begin{tabular}{p{3in} r} % setup 2 columns, first
Boston Bombing Spam/Malware Prevention & Spring 13
\end{tabular}
\begin{itemize} \itemsep -2pt
\item Worked with John Bambenek to analyze a stream of spam related
  to the Boston bombing, analyzed domain registrations, basic malware
  reversing
\item Thanked on isc.sans.org (SANS Internet Storm Center) and featured on WAND local news
\end{itemize}
\begin{tabular}{p{3in} r} % setup 2 columns, first
Quality Evaluation of Obfuscation & Spring 13
\end{tabular}
\begin{itemize} \itemsep -2pt
\item Attempted to develop a metric to allow for the development of
  a framework for analyzing the relative strength of an obfuscation
  routine.  I worked with one other student for a semester project in
  the Network Security course.
\item Presented a poster at Siebel Center in Spring 2013
\end{itemize}
\begin{tabular}{p{3in} r} % setup 2 columns, first
Malware Clustering Script & Fall 12
\end{tabular}
\begin{itemize} \itemsep -2pt
\item Developed a simple malware cluster script written in python using the
  k-means algorithm.  This involved acquiring XML outputs from
  CWSandbox and modeling a feature-set from the results.  I wrote this
  as a final project for a statistics course.
\end{itemize}
\begin{tabular}{p{3in} r} % setup 2 columns, first
UIUC Security CTF Team Presenter & Fall 12
\end{tabular}
\begin{itemize} \itemsep -2pt
\item  Designed and presented a series of lectures on skills required to
  participate in a Security CTF competition
\end{itemize}
\begin{tabular}{p{3in} r} % setup 2 columns, first
Android Telemetry for a Vehicle & Fall 12 - Spring 13
\end{tabular}
\begin{itemize} \itemsep -2pt
\item Developed an android application that communicated with an
  Arduino device connected to a car engine.  This provided real-time
  feedback on car speed, distance traveled, and time elapsed for the
  Ecoillini Shell Marathon Car
\end{itemize}

\section{ACTIVITIES}
\begin{itemize} \itemsep -2pt
\item CSAW Security CTF 2012-2014, played in finals in 2013 for UIUCTF.
\item Various other CTFs (My other favorites are PlaidCTF and GitS)
\item IEEE Hackathon Fall 2012 -- Using OpenCV and an OCR script,
  developed an application that would read in a video feed from a
  webcam and record the Identification Number from an Student ID.
\item ACM Special Interest Group for Security: Member since Spring
  2012, Chair Fall 2012-Spring 2014
\item Undergraduate Computer Science Research Symposium 2012 Poster Presentation
\end{itemize}
\end{resume}
\end{document}
