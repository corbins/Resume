
%%% Local Variables:
%%% mode: latex
%%% TeX-master: t
%%% End:
% LaTeX file for resume
% This file uses the resume document class (res.cls)

\documentclass[margin]{res}
\usepackage{helvetica} % uses helvetica postscript font (download helvetica.sty)
%\usepackage{newcent}   % uses new century schoolbook postscript font
\topmargin=-0.5in  % start text higher on the page
\setlength{\textheight}{10in} % increase text height to fit resume on 1 page
\begin{document}
\name{Corbin Souffrant}

\address{ 507 W Green St. Apt. B, Urbana IL 61801 \\
  Website: www.corbinsouffrant.com \\ Email: souffra2@illinois.edu \\
  Phone: (812) 381-3984
  }


\begin{resume}

\section{OBJECTIVE} Seeking either a full time position or an
internship (My graduation date will be May 2016 if I am receiving an
internship to account for a Masters Degree).  Willing to work in
Computer Security Research, Vulnerability Research and Development, or
Malware Analysis.

\section{EDUCATION}       University of Illinois Urbana-Champaign \\
                Computer Science Engineering B.S., expected May 2014 \\
                G.P.A. 3.28/4.0

                \begin{ncolumn}{2}
                {\bf Relevant Courses} \\
                CS461 - Computer Security I & CS460 - Computer Security Lab \\
                CS498SH - Malware Analysis Lab & CS498LA - Undergraduate Research Lab \\
                CS598MAN - Applied Cryptography & CS467 - Social
                Visualization \\
                CS591RHC - Security Reading Group & CS598MCC - Network
                Security \\
                CS498AL1 - Digital Forensics Lab
		\end{ncolumn}

\section{SKILLS}
\normalsize{\section{Security}}
                 \begin{itemize} \itemsep -2pt
                 \item Static and Dynamic Analysis: IDAPro, OllyDBG,
                   GDB, Wireshark
                 \item Vulnerability Discovery: Fuzzer Development,
                   Source Code Auditing, Testing Framework Design
                 \item Malware Analysis: Unpacking techniques,
                     Anti-debugging + Anti-Reversing removal
                 \end{itemize}
\normalsize{\section{Languages and Libraries}}
	           \begin{itemize} \itemsep -2pt % reduce space between items
                   \item C/C++, Python, x86, MIPS, Java
                   \item ROS (Robot Operating System), OpenCV, Android
                   \end{itemize}

\section{WORK EXPERIENCE}
\begin{tabular}{p{3in} r} % setup 2 columns, first
					    % is 3 inches wide
                  University of Illinois - Researcher &  Fall 13
                  \end{tabular}
                   \begin{itemize} \itemsep -2pt % \item[] prevents a bullet from appearing
                     \item Worked with both Professor Darko Marinov
                       and Professor Matthew Caesar on a new project
                       that aimed to detect incompatibilities between
                       code sources.  Involves analyzing previous
                       literature in the field and coming up with a
                       model to support our project.
		   \end{itemize}
\begin{tabular}{p{3in} r} % setup 2 columns, first
					    % is 3 inches wide
                  University of Illinois - Teaching Assistant&  Fall 13
                  \end{tabular}
                   \begin{itemize} \itemsep -2pt % \item[] prevents a bullet from appearing
                     \item Spent the semester as an Engineering Learning Assistant.
                     \item Helped encourage freshmen in Computer
                       Science to get the most out of their college
                       education.  Designed and Presented lectures as
                       well as acting as a mentor for the students.
		   \end{itemize}
\begin{tabular}{p{3in} r} % setup 2 columns, first
					    % is 3 inches wide
                  Raytheon SIGOVS - Intern &  Summer 13
                  \end{tabular}
                   \begin{itemize} \itemsep -2pt % \item[] prevents a bullet from appearing
                    \item Vulnierability Discovery using a variety of
                      methods.  Developed a smart fuzzer and setup a
                      testing framework.  Found bugs via source code
                      auditing as well.  Worked with browser security
                      and applications on both x86 and ARM architectures.
		   \end{itemize}
		 \begin{tabular}{p{3in} r}
                  University of Illinois - Researcher &  Spring - Summer 12
                 \end{tabular}
		  \begin{itemize} \itemsep -2pt
                   \item Worked with Professor Sam King to design
                     and implement an application framework for
                     general purpose robots.  This involved developing
                     an API to communicate with the robot via a web
                     and android application.  Also wrote applications
                     in python and C++.
                   \item Presented a poster for
                     the research at a research symposium in Siebel in
                     Spring 2012
                  \end{itemize}

\section{PROJECTS}
\begin{tabular}{p{3in} r} % setup 2 columns, first
Automated Malware Analysis & Spring 13
\end{tabular}
\begin{itemize} \itemsep -2pt
\item Set up a Virtual Machine that accepted binaries from a web
  interface.  I then used YARA and CuckooSandbox to process the binary
  and store the results in a database.  I worked with 3 other
  students for a semester project in the Security Lab course.
\end{itemize}
\begin{tabular}{p{3in} r} % setup 2 columns, first
Boston Bombing Spam/Malware Prevention & Spring 13
\end{tabular}
\begin{itemize} \itemsep -2pt
\item Worked with John Bambenek to analyze a stream of spam related
  to the Boston bombing, analyzed domain registrations, basic malware
  reversing
\item Thanked on isc.sans.org (SANS Internet Storm Center) and featured on WAND local news
\end{itemize}
\begin{tabular}{p{3in} r} % setup 2 columns, first
Quality Evaluation of Obfuscation & Spring 13
\end{tabular}
\begin{itemize} \itemsep -2pt
\item Attempted to develop a metric to allow for the development of
  a framework for analyzing the relative strength of an obfuscation
  routine.  I worked with one other student for a semester project in
  the Network Security course.
\item Presented a poster at Siebel Center in Spring 2013
\end{itemize}
\begin{tabular}{p{3in} r} % setup 2 columns, first
Malware Clustering Script & Fall 12
\end{tabular}
\begin{itemize} \itemsep -2pt
\item Developed a simple malware cluster script written in python using the
  k-means algorithm.  This involved acquiring XML outputs from
  CWSandbox and modeling a feature-set from the results.  I wrote this
  as a final project for a statistics course.
\end{itemize}
\begin{tabular}{p{3in} r} % setup 2 columns, first
UIUC Security CTF Team Presenter & Fall 12
\end{tabular}
\begin{itemize} \itemsep -2pt
\item  Designed and presented a series of lectures on skills required to
  participate in a Security CTF competition
\end{itemize}
\begin{tabular}{p{3in} r} % setup 2 columns, first
Android Telemetry for a Vehicle & Fall 12 - Spring 13
\end{tabular}
\begin{itemize} \itemsep -2pt
\item Developed an android application that communicated with an
  Arduino device connected to a car engine.  This provided real-time
  feedback on car speed, distance traveled, and time elapsed for the
  Ecoillini Shell Marathon Car
\end{itemize}

\section{ACTIVITIES}
\begin{itemize} \itemsep -2pt
\item CSAW Security CTF 2012, placed 18th for qualifying teams with
  the ACM Security CTF Team
\item EBay Hackathon Spring 2013 -- Best Use of API for a visualization of product
  sales, with 3 team members.
\item IEEE Hackathon Fall 2012 -- Using OpenCV and an OCR script,
  developed an application that would read in a video feed from a
  webcam and record the Identification Number from an Student ID.
\item ACM Special Interest Group for Security: Member since Spring
  2012, Chair since Fall 2012
\item Undergraduate Computer Science Research Symposium 2012 Poster Presentation
\item Illinois Technology Association Fall Challenge Finalist 2011
\end{itemize}
\end{resume}
\end{document}
